\documentclass[12pt, a4paper]{article}
\title{Not my first LaTeX document}
\author{Illyés Dávid\thanks{Thanks to my Mom}}
\date{\today}
\begin{document}
\maketitle


A jegyzet fő célja, hogy olyan módon adja át a "A Programozás Alapjai 1" nevű tárgy anyagát, hogy az teljesen kezdők számára is könnyen megtanulható legyen.

Ami itt kimarad:
* 1.előadás 2.fejezet - Alapfogalmak
    * Imperatív

Algoritmus:
Gépiesen végrehajtható lépések véges sorozata, amely elvezet a megoldáshoz
Kódolás előtt három dologról kell meggyőződnünk:
* helyes - tényleg azt csinálja amire szükségünk van
* teljes _ minden elhetséges esetben elvégzi a dolgát
* véges - véges sok lépésben befejeződik

Algoritmusok leírása:

Algoritmusok leírási nyelvfüggetlen leírási módja a pszeudokód (álkód), ez természetes nyelven, de precÍzen megfogalmazott utasítássorozatot jelent.

## IDE JÖHET EGY EGYSZERŰ PÉLDA ##

Algoritmusok grafikus ábrázolásának eszköze a folyamatábra
Egy be- és kimenetű program folyamatábrája START és STOP  elemek között helyezkedik el

## IDE KELL EGY ÁBRA ##

A folyamatábra az alábbi elemekből épül fel

## IDE KELL EGY ÁBRA ##

## IDE JÖHET EGY EGYSZERŰ PÉLDA ##

## IDE JÖHET EGY EGYSZERŰ PÉLDA CIKLUSOKRA ##

Az algoritmus adatokon, adatokkal dolgozik

Adat: Az adat minden, amit a külvilágból számítógépünkben leképezve tárolunk

Az adatnak van típusa (ez lehet szám, betű, szín) és értéke. Az adat típusa meghatározza az adat értékkészletét és az adaton végezhető műveleteket


\b
    *egin{atívitefolyamatámize}
    \item [\textbf{TO-DO list:}]
    \item rendező algoritmusokat felsorolni

\end{itemize}

\end{document}



